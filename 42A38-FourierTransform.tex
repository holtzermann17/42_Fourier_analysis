\documentclass[12pt]{article}
\usepackage{pmmeta}
\pmcanonicalname{FourierTransform}
\pmcreated{2013-03-22 12:34:28}
\pmmodified{2013-03-22 12:34:28}
\pmowner{mathwizard}{128}
\pmmodifier{mathwizard}{128}
\pmtitle{Fourier transform}
\pmrecord{17}{32823}
\pmprivacy{1}
\pmauthor{mathwizard}{128}
\pmtype{Definition}
\pmcomment{trigger rebuild}
\pmclassification{msc}{42A38}
\pmrelated{Wavelet}
\pmrelated{ProgressiveFunction}
\pmrelated{DiscreteFourierTransform}
\pmrelated{FourierSeriesInComplexFormAndFourierIntegral}
\pmrelated{TwoDimensionalFourierTransforms}
\pmrelated{TableOfGeneralizedFourierAndMeasuredGroupoidTransforms}
\pmdefines{first Parseval theorem}

\endmetadata

% this is the default PlanetMath preamble.  as your knowledge
% of TeX increases, you will probably want to edit this, but
% it should be fine as is for beginners.

% almost certainly you want these
\usepackage{amssymb}
\usepackage{amsmath}
\usepackage{amsfonts}

% used for TeXing text within eps files
%\usepackage{psfrag}
% need this for including graphics (\includegraphics)
%\usepackage{graphicx}
% for neatly defining theorems and propositions
%\usepackage{amsthm}
% making logically defined graphics
%%%\usepackage{xypic}

% there are many more packages, add them here as you need them

% define commands here
\begin{document}
The \emph{Fourier transform} $F(s)$ of a function $f(t)$ is defined as follows:

$$ F(s)=\frac{1}{\sqrt{2\pi}}\int_{-\infty}^{\infty}e^{-ist}f(t)dt.$$

The Fourier transform exists if $f$ is Lebesgue integrable on the whole real axis.

If $f$ is Lebesgue integrable and can be divided into a finite number of continuous, monotone functions and at every point both one-sided limits exist, the Fourier transform can be inverted:

$$f(t)=\frac{1}{\sqrt{2\pi}}\int_{-\infty}^{\infty}e^{ist}F(s)ds.$$

Sometimes the Fourier transform is also defined without the factor $\frac{1}{\sqrt{2\pi}}$ in one direction, but therefore giving the transform into the other direction a factor $\frac{1}{2\pi}$. So when looking a transform up in a table you should find out how it is defined in that table.

The Fourier transform has some important properties, that can be used when solving differential equations. We denote the Fourier transform of $f$ with respect to $t$ in terms of $s$ by $\mathcal{F}_t(f)$.
\begin{itemize}
\item $\mathcal{F}_t(af+bg)=a\mathcal{F}_t(f)+b\mathcal{F}_t(g),$\\
where $a$ and $b$ are constants.
\item $\mathcal{F}_t\left(\frac{\partial}{\partial t}f\right)=is\mathcal{F}_t(f).$
\item $\mathcal{F}_t\left(\frac{\partial}{\partial x}f\right)=\frac{\partial}{\partial x}\mathcal{F}_t(f).$
\item We define the bilateral convolution of two functions $f_1$ and $f_2$ as: 
$$(f_1\ast f_2)(t):=\frac{1}{\sqrt{2\pi}}\int_{-\infty}^{\infty}f_1(\tau)f_2(t-\tau)d\tau.$$
Then the following equation holds:
$$\mathcal{F}_t((f_1\ast f_2)(t))=\mathcal{F}_t(f_1)\cdot\mathcal{F}_t(f_2).$$
\end{itemize}
If $f(t)$ is some signal (maybe a \PMlinkescapeword{sound} wave) then the frequency domain of $f$ is given as $\mathcal{F}_t(f)$. Rayleigh's theorem states that then the energy $E$ carried by the signal $f$ given by:
$$E=\int_{-\infty}^{\infty}|f(t)|^2dt$$
can also be expressed as:
$$E=\int_{-\infty}^{\infty}|\mathcal{F}_t(f)(s)|^2ds.$$
In general we have:
$$\int_{-\infty}^{\infty}|f(t)|^2dt=\int_{-\infty}^{\infty}|\mathcal{F}_t(f)(s)|^2ds,$$
also known as the \emph{first Parseval theorem}.
%%%%%
%%%%%
\end{document}
