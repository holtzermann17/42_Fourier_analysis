\documentclass[12pt]{article}
\usepackage{pmmeta}
\pmcanonicalname{ParsevalEquality}
\pmcreated{2013-03-22 13:57:10}
\pmmodified{2013-03-22 13:57:10}
\pmowner{asteroid}{17536}
\pmmodifier{asteroid}{17536}
\pmtitle{Parseval equality}
\pmrecord{11}{34717}
\pmprivacy{1}
\pmauthor{asteroid}{17536}
\pmtype{Theorem}
\pmcomment{trigger rebuild}
\pmclassification{msc}{42B05}
\pmsynonym{Parseval equation}{ParsevalEquality}
\pmsynonym{Parseval identity}{ParsevalEquality}
\pmsynonym{Lyapunov equation}{ParsevalEquality}
\pmrelated{BesselInequality}
\pmrelated{ValueOfTheRiemannZetaFunctionAtS2}
\pmdefines{Parseval theorem}

\endmetadata

% this is the default PlanetMath preamble.  as your knowledge
% of TeX increases, you will probably want to edit this, but
% it should be fine as is for beginners.

% almost certainly you want these
\usepackage{amssymb}
\usepackage{amsmath}
\usepackage{amsfonts}

% used for TeXing text within eps files
%\usepackage{psfrag}
% need this for including graphics (\includegraphics)
%\usepackage{graphicx}
% for neatly defining theorems and propositions
%\usepackage{amsthm}
% making logically defined graphics
%%%\usepackage{xypic}

% there are many more packages, add them here as you need them

% define commands here
\begin{document}
\subsection{Parseval's Equality}

{\bf Theorem. --} If $\{e_j\!:\; j \in J \}$ is an orthonormal basis of an Hilbert space $H$, then for every $x \in H$ the following equality holds:
\begin{displaymath}
\|x\|^2 = \sum_{j \in J} | \langle x , e_j \rangle |^2 .
\end{displaymath}

The above theorem is a more sophisticated form of Bessel's inequality (where the inequality is in fact an equality). The difference is that for Bessel's inequality it is only required that the set $\{e_j : j \in J \}$ is an orthonormal set, not necessarily an orthonormal basis.

\subsection{Parseval's Theorem}

Applying Parseval's equality on the Hilbert space \PMlinkname{$L^2([-\pi,\pi])$}{LpSpace}, of square integrable functions on the interval $[-\pi,\pi]$, with the orthonormal basis consisting of trigonometric functions, we obtain

{\bf Theorem (Parseval's theorem). --} Let $f$ be a Riemann square integrable function from $[-\pi,\pi]$ to $\mathbb{R}$.\, The following equality holds
$$\frac{1}{\pi}\int_{-\pi}^{\pi}f^2(x)dx = \frac{(a_0^f)^2}{2} + \sum_{k=1}^{\infty}[(a_k^f)^2+(b_k^f)^2],$$
where $a_0^f$, $a_k^f$, $b_k^f$ are the Fourier coefficients of the function $f$.

The function $f$ can be a Lebesgue-integrable function, if we use the Lebesgue integral in place of the Riemann integral.
%%%%%
%%%%%
\end{document}
