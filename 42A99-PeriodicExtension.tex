\documentclass[12pt]{article}
\usepackage{pmmeta}
\pmcanonicalname{PeriodicExtension}
\pmcreated{2013-03-22 17:33:00}
\pmmodified{2013-03-22 17:33:00}
\pmowner{CWoo}{3771}
\pmmodifier{CWoo}{3771}
\pmtitle{periodic extension}
\pmrecord{17}{39955}
\pmprivacy{1}
\pmauthor{CWoo}{3771}
\pmtype{Definition}
\pmcomment{trigger rebuild}
\pmclassification{msc}{42A99}
\pmrelated{PeriodicFunctions}
\pmrelated{TriangularWaveFunction}
\pmrelated{SawBladeFunction}

\endmetadata

\usepackage{amssymb,amscd}
\usepackage{amsmath}
\usepackage{amsfonts}
\usepackage{mathrsfs}

% used for TeXing text within eps files
%\usepackage{psfrag}
% need this for including graphics (\includegraphics)
%\usepackage{graphicx}
% for neatly defining theorems and propositions
\usepackage{amsthm}
% making logically defined graphics
%%\usepackage{xypic}
\usepackage{pst-plot}
\usepackage{psfrag}

% define commands here
\newtheorem{prop}{Proposition}
\newtheorem{thm}{Theorem}
\newtheorem{ex}{Example}
\newcommand{\real}{\mathbb{R}}
\newcommand{\pdiff}[2]{\frac{\partial #1}{\partial #2}}
\newcommand{\mpdiff}[3]{\frac{\partial^#1 #2}{\partial #3^#1}}
\begin{document}
Let $f$ be a function defined on some real interval $[a,b]$.  By a \emph{periodic extension} of $f$ to the real line we mean a function $g$ such that 
\begin{enumerate}
\item $g$ is defined on $\mathbb{R}$ except perhaps at points $a+n(b-a)$, where $n\in\mathbb{Z}$;
\item $g(x)=f(x)$ for all $x\in (a,b)$, and
\item $g(x+n(a-b))=g(x)$ for all $x\in (a,b)$ and all integers $n$.
\end{enumerate}

The best way to understand periodic extensions of a function is to look the graph of a periodic extension of a real-valued function.  For example, let $f(x)=x$ be defined on $[-1,1]$.  The graph of $f$ looks like 
\begin{center}
\begin{pspicture}(-6,-2)(6,2)
\psaxes[Dx=9,Dy=2]{->}(0,0)(-5.5,-1.5)(5.5,1.5)
\rput(5.8,0){$x$}
\rput(0.2,1.6){$y$}
\psline(-1,-1)(1,1)
\psdots[dotscale=1](1,1)(-1,-1)
\end{pspicture}
\end{center}

Then the graph a periodic extension $g$ of $f$ may look like
\begin{center}
\begin{pspicture}(-6,-2)(6,2)
\psaxes[Dx=9,Dy=2]{->}(0,0)(-5.5,-1.5)(5.5,1.5)
\rput(5.8,0){$x$}
\rput(0.2,1.6){$y$}
\psline(-5.5,0.5)(-5,1)
\psline(-5,-1)(-3,1)
\psline(-3,-1)(-1,1)
\psline(-1,-1)(1,1)
\psline(1,-1)(3,1)
\psline(3,-1)(5,1)
\psline(5,-1)(5.5,-0.5)
\psdots[dotstyle=o,dotscale=1](-1,-1)(-3,-1)(1,-1)(-5,-1)(3,-1)(5,-1)
\psdots[dotscale=1](1,1)(-1,1)(3,1)(-3,1)(5,1)(-5,1)
\end{pspicture}
\end{center}

or look like

\begin{center}
\begin{pspicture}(-5.5,-2)(5.5,2)
\psaxes[Dx=9,Dy=2]{->}(0,0)(-5.5,-1.5)(5.5,1.5)
\rput(5.8,0){$x$}
\rput(0.2,1.6){$y$}
\psline(-5.5,0.5)(-5,1)
\psline(-5,-1)(-3,1)
\psline(-3,-1)(-1,1)
\psline(-1,-1)(1,1)
\psline(1,-1)(3,1)
\psline(3,-1)(5,1)
\psline(5,-1)(5.5,-0.5)
\psdots[dotstyle=o,dotscale=1](-1,-1)(1,1)(-3,-1)(-1,1)(1,-1)(3,1)(-5,-1)(-3,1)(3,-1)(5,1)(-5,1)(5,-1)
\psdots[dotscale=1](-1,0)(1,0)(-3,0)(3,0)(-5,0)(5,0)
\end{pspicture}
\end{center}

Notice the two periodic extensions of $f$ are identical except at odd integer points on the $x$-axis.  The reason why we do not require $g$ to agree with $f$ on the end points of $[a,b]$ is because we do not know if $f(a)=f(b)$.  If they do not agree, requiring that $g=f$ on all of $[a,b]$ may result in points $a+n(b-a)$ getting mapped to two distinct values $f(a)$ and $f(b)$, rendering $g$ not well-defined.  In fact, if $f$ does not agree on its endpoints, no periodic extensions of $f$ are continuous.

Notice, also, that the domain of function $f$ does not have to be the entire closed interval $[a,b]$.  The domain of $f$ may very well be a subset $S\subseteq [a,b]$.  For example, $f(x)=x$ may be a function defined on the open interval $(-1,1)$.  The two graphs above are again graphs of periodic extensions of $f$.  

However, if $S$ is a proper subset of $[a,b]$ that is not the open interval $(a,b)$, then the definition of a periodic extension needs to be modified: $g$ is a periodic extension of $f$ defined on $S\subseteq [a,b]$ if 
\begin{enumerate}
\item $g$ is defined on a subset $T\subseteq \mathbb{R}$ except perhaps at points $a+n(b-a)$, where $T=\lbrace x+n(a-b)\mid x\in S\rbrace$ and $n\in\mathbb{Z}$;
\item $g(x)=f(x)$ for all $x\in S-\lbrace a,b\rbrace$, and
\item $g(x+n(a-b))=g(x)$ for all $x\in S-\lbrace a,b\rbrace$ and all integers $n$.
\end{enumerate}
We generally assume that $a=\inf S$ and $b=\sup S$.

For example, if $f(x)=x$ for all rational numbers $x\in [-1,1]$, then a periodic extension of $f$ has its domain the set of all rational numbers except perhaps at are odd integers.

\textbf{Remarks}.
\begin{itemize}
\item
Trigonometric functions defined on $\mathbb{R}$ are periodic extensions of the trigonometric functions defined for angles in the interval $[0,2\pi]$.
\item
Suppose $f$ is defined either on a closed interval $[a,b]$ or an open interval $(a,b)$, $f$ has a continuous periodic extension (defined on all of $\mathbb{R}$) iff $f$ is continuous and that 
\begin{enumerate}
\item
either $f(a)=f(b)$ when $f$ is defined on a closed interval, or 
\item
$f(a+)=f(b-)$ when $f$ is defined on an open interval, where $f(a+)$ is the one-sided limit approaching $a$ from the right, and $f(b-)$ is the one-sided limit approaching $b$ from the left.
\end{enumerate}
Simply define the periodic extension $g$ so that either $g(a)=f(a)$, or $g(a)=f(a+)$.  For example, the following graph
\begin{center}
\begin{pspicture}(-6,-0.5)(6,2)
\psaxes[Dx=9,Dy=2]{->}(0,0)(-5.5,-0.5)(5.5,1.5)
\rput(5.8,0){$x$}
\rput(0.2,1.6){$y$}
\psline(-5.5,0.5)(-5,1)
\psline(-5,1)(-4,0)
\psline(-4,0)(-3,1)
\psline(-3,1)(-2,0)
\psline(-2,0)(-1,1)
\psline(-1,1)(0,0)
\psline(0,0)(1,1)
\psline(1,1)(2,0)
\psline(2,0)(3,1)
\psline(3,1)(4,0)
\psline(4,0)(5,1)
\psline(5,1)(5.5,0.5)
\end{pspicture}
\end{center}
is the graph of the continuous periodic extension of a function $f_1$ given by $f_1(x)=|x|$ defined on $(-1,1)$, or a function $f_2$ defined on $(0,2)$, given by $f_2(x)=x$ for $0<x\le 1$ and $f_2(x)=2-x$ for $1\le x<2$.  With $f_1$, we see that $f_1(-1+)=f_1(1-)=1$, while with $f_2$, we have $f_2(0+)=f_2(2-)=0$.

It is easy to see that if a continuous periodic extension of a function exists, then it is unique.
\item
Higher dimensional periodic extensions may also be defined for functions defined on a parallelepiped ($n$-dimensional analog of a parallelogram).  A periodic extension $g$ of a function $f$ defined on a parallelepiped is a function such that its projection $p_i(g)$ onto axis $i$ in $\mathbb{R}^n$ is a periodic extension of the projection $p_i(f)$ of $f$ onto axis $i$.
\end{itemize}

\begin{thebibliography}{9}
\bibitem{gt} G.P. Tolstov, \textsl{Fourier Series}, Prentice-Hall, 1962.
\end{thebibliography}
%%%%%
%%%%%
\end{document}
