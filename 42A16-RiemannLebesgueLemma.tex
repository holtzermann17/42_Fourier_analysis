\documentclass[12pt]{article}
\usepackage{pmmeta}
\pmcanonicalname{RiemannLebesgueLemma}
\pmcreated{2013-03-22 13:08:04}
\pmmodified{2013-03-22 13:08:04}
\pmowner{rmilson}{146}
\pmmodifier{rmilson}{146}
\pmtitle{Riemann-Lebesgue lemma}
\pmrecord{5}{33571}
\pmprivacy{1}
\pmauthor{rmilson}{146}
\pmtype{Theorem}
\pmcomment{trigger rebuild}
\pmclassification{msc}{42A16}

\endmetadata

\usepackage{amsmath}
\usepackage{amsfonts}
\usepackage{amssymb}
\newcommand{\reals}{\mathbb{R}}
\newcommand{\natnums}{\mathbb{N}}
\newcommand{\cnums}{\mathbb{C}}
\newcommand{\znums}{\mathbb{Z}}
\newcommand{\lp}{\left(}
\newcommand{\rp}{\right)}
\newcommand{\lb}{\left[}
\newcommand{\rb}{\right]}
\newcommand{\supth}{^{\text{th}}}
\newtheorem{proposition}{Proposition}
\newtheorem{definition}[proposition]{Definition}

\newtheorem{theorem}[proposition]{Theorem}
\begin{document}
\noindent
{\bf \PMlinkescapetext{Proposition}.}
Let $f:[a,b]\rightarrow\cnums$ be a measurable function.  If $f$ is
$L^1$ integrable, that is to say if the Lebesgue integral of $|f|$ is
finite, then
$$\int^b_a f(x) e^{inx}\,dx \rightarrow 0,\quad{as}\quad n\rightarrow
\pm\infty.$$


The above result, commonly known as the Riemann-Lebesgue lemma, is of
basic importance in harmonic analysis.  It is equivalent to the
assertion that the Fourier coefficients $\hat{f}_n$ of a periodic, integrable
function $f(x)$, tend to $0$ as $n\rightarrow \pm\infty$.

The proof can be organized into 3 steps.

\noindent
\emph{Step 1.}  An elementary calculation shows that 
$$\int_I e^{inx}\,dx \rightarrow 0,\quad{as}\quad n\rightarrow
\pm\infty$$
for every interval $I\subset[a,b]$. The proposition is therefore true
for all step functions with \PMlinkname{support}{SupportOfFunction} in $[a,b]$.

\noindent
\emph{Step 2.}  
By the monotone convergence theorem, the proposition is true for all
positive functions, integrable on $[a,b]$.

\noindent
\emph{Step 3.}  Let $f$ be an arbitrary measurable function,
integrable on $[a,b]$.  The proposition is true for such a general
$f$, because one can always write 
$$f=g-h,$$
where $g$ and $h$ are positive functions, integrable on
$[a,b]$.
%%%%%
%%%%%
\end{document}
