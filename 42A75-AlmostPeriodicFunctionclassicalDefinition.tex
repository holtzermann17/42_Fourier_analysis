\documentclass[12pt]{article}
\usepackage{pmmeta}
\pmcanonicalname{AlmostPeriodicFunctionclassicalDefinition}
\pmcreated{2013-03-22 14:53:14}
\pmmodified{2013-03-22 14:53:14}
\pmowner{drini}{3}
\pmmodifier{drini}{3}
\pmtitle{almost periodic function (classical definition)}
\pmrecord{13}{36567}
\pmprivacy{1}
\pmauthor{drini}{3}
\pmtype{Definition}
\pmcomment{trigger rebuild}
\pmclassification{msc}{42A75}
\pmsynonym{almost periodic function}{AlmostPeriodicFunctionclassicalDefinition}
\pmrelated{ExampleOfNonSeperableHilbertSpace}
\pmdefines{almost periodic}

% this is the default PlanetMath preamble.  as your knowledge
% of TeX increases, you will probably want to edit this, but
% it should be fine as is for beginners.

% almost certainly you want these
\usepackage{amssymb}
\usepackage{amsmath}
\usepackage{amsfonts}

% used for TeXing text within eps files
%\usepackage{psfrag}
% need this for including graphics (\includegraphics)
%\usepackage{graphicx}
% for neatly defining theorems and propositions
%\usepackage{amsthm}
% making logically defined graphics
%%%\usepackage{xypic}

% there are many more packages, add them here as you need them

% define commands here
\begin{document}
A continuous function $f \colon \mathbb{R} \to \mathbb{R}$ is said to be \emph{almost periodic} if, for every $\epsilon > 0$, there exists an a number $L_\epsilon > 0$ such that for every interval $I$ of length $L_\epsilon$ there exists a number $\omega_I \in I$ such that 
 $$| f(x + \omega_I) - f(x) | < \epsilon$$
whenever $x \in \mathbb{R}$. 

Intuition: we want the function to have an "approximate period".
However, it is easy to write too weak condition. First, we want
uniform estimate in $x$. If we allow $ \omega$ to be small than
the condition degenerates to uniform continuity. If we require
a single $ \omega$, than the condition still is too weak (it
allows pretty wide changes). For periodic function every multiple
of a period is still a period. So, if the length of an interval
is longer than the period, then the interval contains a period.
The definition of almost periodic functions mimics the above
property of periodic functions: every sufficiently long interval
should contain an approximate period.

It is possible to generalize this notion.  The range of the function can be taken to be a normed vector space --- in the first definition, we merely need to replace the absolute value with the norm:
 $$\| f(x + \omega) - f(x) \| < \epsilon$$
In the second definition, interpret uniform convergence as uniform convergence with respect to the norm.  A common case of this is the case where the range is the complex numbers.  It is worth noting that if the vector space is finite dimensional, a function is almost periodic if and only if each of its components with respect to a basis is almost periodic.

Also the domain may be taken to be a group $G$.  A function is called almost periodic iff set of its translates is pre-compact (compact after completion).
Equivalently, a continuous function $f$ on a topological group $G$ is
almost periodic iff there is a compact group $K$, a continuous
function $g$ on $K$ and a (continuous) homomorphism $h$ form $G$ 
to $K$ such that $f$ is the composition of $g$ and $h$. 
 The classical case described above arises when the group is the additive group of the real number field.  Almost periodic functions with respect to groups play a role in the representation theory of non-compact Lie algebras.  (In the compact case, they are trivial --- all continuous functions are almost periodic.)

The notion of an almost periodic function should not be confused with the notion of quasiperiodic function.
%%%%%
%%%%%
\end{document}
