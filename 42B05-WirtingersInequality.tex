\documentclass[12pt]{article}
\usepackage{pmmeta}
\pmcanonicalname{WirtingersInequality}
\pmcreated{2013-03-22 14:02:38}
\pmmodified{2013-03-22 14:02:38}
\pmowner{rspuzio}{6075}
\pmmodifier{rspuzio}{6075}
\pmtitle{Wirtinger's inequality}
\pmrecord{9}{35393}
\pmprivacy{1}
\pmauthor{rspuzio}{6075}
\pmtype{Theorem}
\pmcomment{trigger rebuild}
\pmclassification{msc}{42B05}
\pmsynonym{Wirtinger inequality}{WirtingersInequality}
%\pmkeywords{parseval}
%\pmkeywords{isoperimetric}

\endmetadata

\usepackage{amssymb}
\usepackage{amsmath}
\usepackage{amsfonts}
\begin{document}
\newcommand{\R}{\mathbb{R}}
\PMlinkescapeword{proof}
\PMlinkescapeword{identity}

\textbf{Theorem: }
Let $f\colon\R\to\R$ be a periodic function of period $2\pi$, which is
continuous and has a continuous derivative throughout $\R$, and such
that
\begin{equation} \label{eq:1}
\int_0^{2\pi}f(x)=0\;.
\end{equation}
Then
\begin{equation} \label{eq:2}
\int_0^{2\pi}f'^2(x)dx\ge\int_0^{2\pi}f^2(x)dx
\end{equation}
with equality if and only if $f(x)=a\cos x+b\sin x$ for some $a$ and $b$
(or equivalently $f(x)=c\sin (x+d)$ for some $c$ and $d$).

\textbf{Proof: }Since Dirichlet's conditions are met, we
can write
$$f(x)=\frac{1}{2}a_0+\sum_{n\ge 1}(a_n\sin nx+b_n\cos nx)$$
and moreover $a_0=0$ by \eqref{eq:1}. By Parseval's identity,
$$\int_0^{2\pi}f^2(x)dx=\sum_{n=1}^\infty(a_n^2+b_n^2)$$
and
$$\int_0^{2\pi}f'^2(x)dx=\sum_{n=1}^\infty n^2(a_n^2+b_n^2)$$
and since the summands are all $\ge 0$, we get \eqref{eq:2},
with equality if and only if $a_n=b_n=0$ for all $n\ge 2$.

Hurwitz used Wirtinger's inequality in his tidy 1904
proof of the isoperimetric inequality.
%%%%%
%%%%%
\end{document}
