\documentclass[12pt]{article}
\usepackage{pmmeta}
\pmcanonicalname{EveryNormedSpaceWithSchauderBasisIsSeparable}
\pmcreated{2013-03-22 17:36:07}
\pmmodified{2013-03-22 17:36:07}
\pmowner{asteroid}{17536}
\pmmodifier{asteroid}{17536}
\pmtitle{every normed space with Schauder basis is separable}
\pmrecord{11}{40016}
\pmprivacy{1}
\pmauthor{asteroid}{17536}
\pmtype{Theorem}
\pmcomment{trigger rebuild}
\pmclassification{msc}{42-00}
\pmclassification{msc}{15A03}
\pmclassification{msc}{46B15}
%\pmkeywords{separable}
%\pmkeywords{Schauder basis}
%\pmkeywords{normed vector space}
%\pmkeywords{normed space}
%\pmkeywords{Banach space}

\endmetadata

\usepackage{amssymb}
\usepackage{amsmath}
\usepackage{amsfonts}

\begin{document}
\PMlinkescapeword{completes}
\PMlinkescapeword{dense}
\PMlinkescapeword{density}
\PMlinkescapeword{modification}
\PMlinkescapephrase{dense in}

Here we show that every normed space that has a Schauder basis is
\PMlinkname{separable}{Separable}.
Note that we are (implicitly) assuming that the normed spaces in question
are spaces over the field $K$ where $K$ is either $\mathbb{R}$ or $\mathbb{C}$.
So let $(X, \left\|\cdot\right\|)$ be a normed space with Schauder basis,
say $S = \left\{e_1, e_2,\dots\right\}$.
Notice that our notation implies that $S$ is infinite.
In finite dimensional case,
the same proof with a slight modification will yield the result.

Now, set $Q$ to be the set of all finite sums $q_1e_1+\cdots + q_ne_n$
such that each $q_j = a_j + b_ji$ where $a_j, b_j \in \mathbb{Q}$.
Clearly $Q$ is countable.
It remains to show that $Q$ is \PMlinkname{dense}{Dense} in $X$.

Let $\epsilon > 0$. Let $x \in X$.
By definition of Schauder basis,
there is a sequence of scalars $(\alpha_n)$
and there exists $N$ such that for all $n \geq N$ we have,
$$\left\|\sum_{j=1}^{n}\alpha_je_j - x\right\| < \epsilon/2$$
But then in particular,
$$\left\|\sum_{j=1}^{N}\alpha_je_j - x\right\| < \epsilon/2$$
Furthermore, by density of $\mathbb{Q}$ in $\mathbb{R}$,
we know that there exist constants
$a_1,\dots, a_N, b_1,\dots, b_N$ in $\mathbb{Q}$ such that,
$$\left\|\sum_{j=1}^N(a_j + b_ji)e_j - \sum_{j=1}^N\alpha_je_j\right\|<\epsilon/2$$
By triangle inequality we obtain:
$$\left\|\sum_{j=1}^N(a_j + b_ji)e_j - x\right\| \leq \left\|\sum_{j=1}^N(a_j + b_ji)e_j - \sum_{j=1}^N\alpha_je_j\right\| + \left\|\sum_{j=1}^{N}\alpha_je_j - x\right\| < \epsilon$$
Noting that
$$\sum_{j=1}^N(a_j + b_ji)e_j$$ is an element of $Q$ (by construction of $Q$)
and that $x$ and $\epsilon$ were arbitrary,
we conclude that every neighborhood of $x$ contains an element of $Q$,
for all $x$ in $X$.
This proves that $Q$ is dense in $X$ and completes the proof.
%%%%%
%%%%%
\end{document}
