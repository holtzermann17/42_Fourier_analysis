\documentclass[12pt]{article}
\usepackage{pmmeta}
\pmcanonicalname{PlancherelsTheorem}
\pmcreated{2013-03-22 16:29:00}
\pmmodified{2013-03-22 16:29:00}
\pmowner{stevecheng}{10074}
\pmmodifier{stevecheng}{10074}
\pmtitle{Plancherel's theorem}
\pmrecord{11}{38651}
\pmprivacy{1}
\pmauthor{stevecheng}{10074}
\pmtype{Theorem}
\pmcomment{trigger rebuild}
\pmclassification{msc}{42B10}
\pmclassification{msc}{42A38}
\pmrelated{ProofOfSamplingTheorem}

\endmetadata

\usepackage{amssymb}
\usepackage{amsmath}
\usepackage{amsfonts}
\usepackage{amsthm}
\usepackage{enumerate}
%\usepackage{graphicx}
%\usepackage{psfrag}
%%%\usepackage{xypic}

% define commands here
\newcommand{\complex}{\mathbb{C}}
\newcommand{\real}{\mathbb{R}}
\newcommand{\rat}{\mathbb{Q}}
\newcommand{\nat}{\mathbb{N}}

\newcommand{\Le}{\mathbf{L}}
\newcommand{\FT}{\mathcal{F}}
\newcommand{\indc}{\mathbb{I}}

\providecommand{\abs}[1]{\lvert#1\rvert}
\providecommand{\absW}[1]{\left\lvert#1\right\rvert}
\providecommand{\absB}[1]{\Bigl\lvert#1\Bigr\rvert}
\providecommand{\norm}[1]{\lVert#1\rVert}
\providecommand{\normW}[1]{\left\lVert#1\right\rVert}
\providecommand{\normB}[1]{\Bigl\lVert#1\Bigr\rVert}
\providecommand{\defnterm}[1]{\emph{#1}}


\begin{document}
\PMlinkescapeword{states}
\PMlinkescapeword{terms}
\PMlinkescapeword{similar}

\subsection{Statement of theorem}

\emph{Plancherel's Theorem} states that 
the unitary Fourier transform of $\Le^1$ functions 
(the \PMlinkname{Lebesgue-integrable functions}{Integral3}) on $\real^n$
extends to a unitary isomorphism on $\Le^2$ (the square-integrable functions).

Thus, the following two fundamental properties hold
for the Fourier transform $\FT$ on 
$\Le^2$ functions $g \colon \real^n \to \complex$:
\begin{enumerate}[i]
\item
\[
\FT^{-1} (\FT g) = g = \FT( \FT^{-1} g )\,.
\]
The equalities are as elements of $\Le^2$; in terms of pointwise functions,
the equalities hold almost everywhere on $\real^n$.
\item
The Fourier transform preserves $\Le^2$ norms:
\[
\int_{\real^n} \abs{\FT g(\xi)}^2 \, d\xi
= \norm{ \FT g }_{\Le^2}^2 = \norm{g}_{\Le^2}^2
= \int_{\real^n} \abs{g(x)}^2 \, dx\,.
\]
\end{enumerate}

\subsection{Extension of the Fourier transform to $\Le^2$}

The extension $\FT$ of the usual 
Fourier transform can be described concretely as follows: given a $\Le^2$ function $g\colon \real^n \to \complex$,
take any sequence $g_k \colon \real^n \to \complex$
of $\Le^1$ functions that converge in $\Le^2$ to $g$.
The Fourier transforms
\[
\FT g_k (\xi) = \int_{\real^n} g_k(x) \, e^{-2\pi i \xi \cdot x} \, dx\,,
\quad \xi \in \real^n
\]
are defined as usual,
and $\FT g$ can be obtained as the $\Le^2$ limit 
of $\FT g_k$.

In the one-dimensional case, a 
common sequence of approximating sequences to take is
$g_k = g \cdot \indc_{[-k,k]}$; in that case we have
\[
\FT g(\xi) = \lim_{T \to \infty} \int_{-T}^T g(t) \, e^{-2\pi i \xi t} \, dt\,,
\quad \xi \in \real\,.
\]

The inverse Fourier transform $\FT^{-1}$ can be obtained in a similar way
to $\FT$, using approximating functions $g_k$:
\[
\FT^{-1} g_k (x) = \int_{\real^n} g_k(\xi) \, e^{2\pi i \xi \cdot x} \, dx\,,
\quad x \in \real^n\,.
\]

\subsection{Note on different conventions}

Here, we have used the convention for the Fourier transform $\FT$ that $\xi$ denotes ``ordinary frequency'', i.e. the exponential contains factors of $2\pi$.  Another common convention has $\xi$ replaced by $\omega$ denoting the ``angular frequency'', with factors $2\pi$ occurring not in the exponent, but as multiplicative constants. In this case property (i) above still holds,
but property (ii) will not hold unless the multiplicative constants
in front of the forward and inverse Fourier transform are chosen properly.

\begin{thebibliography}{3}
\bibitem[Folland]{Folland}
Gerald B. Folland. \emph{Real Analysis: Modern Techniques and Their Applications}, second ed. Wiley-Interscience, 1999.

\bibitem[Katznelson]{Katznelson}
Yitzhak Katznelson. \emph{An Introduction to Harmonic Analysis}, second ed. Dover Publications, 1976.

\bibitem[Wiki]{Wiki}
``
\PMlinkexternal{Fourier transform}{http://en.wikipedia.org/wiki/Continuous_Fourier_transform}
'', \emph{Wikipedia, The Free Encyclopedia}.  Accessed 22 December, 2006.

\end{thebibliography}


%%%%%
%%%%%
\end{document}
