\documentclass[12pt]{article}
\usepackage{pmmeta}
\pmcanonicalname{FourierSeriesInComplexFormAndFourierIntegral}
\pmcreated{2013-03-22 18:02:54}
\pmmodified{2013-03-22 18:02:54}
\pmowner{pahio}{2872}
\pmmodifier{pahio}{2872}
\pmtitle{Fourier series in complex form and Fourier integral}
\pmrecord{10}{40573}
\pmprivacy{1}
\pmauthor{pahio}{2872}
\pmtype{Derivation}
\pmcomment{trigger rebuild}
\pmclassification{msc}{42A38}
\pmclassification{msc}{42A16}
\pmclassification{msc}{44A55}
%\pmkeywords{Fourier series}
\pmrelated{FourierTransform}
\pmrelated{KalleVaisala}
\pmdefines{Fourier integral}

\endmetadata

% this is the default PlanetMath preamble.  as your knowledge
% of TeX increases, you will probably want to edit this, but
% it should be fine as is for beginners.

% almost certainly you want these
\usepackage{amssymb}
\usepackage{amsmath}
\usepackage{amsfonts}

% used for TeXing text within eps files
%\usepackage{psfrag}
% need this for including graphics (\includegraphics)
%\usepackage{graphicx}
% for neatly defining theorems and propositions
 \usepackage{amsthm}
% making logically defined graphics
%%%\usepackage{xypic}

% there are many more packages, add them here as you need them

% define commands here

\theoremstyle{definition}
\newtheorem*{thmplain}{Theorem}

\begin{document}
\PMlinkescapeword{expansion} \PMlinkescapeword{harmonics}

\subsection{Fourier series in complex form}

The Fourier series expansion of a Riemann integrable real function $f$ on the interval \,$[-p,\,p]$\, is
\begin{align}
f(t) = \frac{a_0}{2}+\sum_{n=1}^\infty\left(a_n\cos{\frac{n\pi t}{p}}+b_n\sin{\frac{n\pi t}{p}}\right),
\end{align}
where the coefficients are
\begin{align}
a_n = \frac{1}{p}\int_{-p}^{\,p}f(x)\cos{\frac{n\pi t}{p}}\,dt, \quad
b_n = \frac{1}{p}\int_{-p}^{\,p}f(x)\sin{\frac{n\pi t}{p}}\,dt.
\end{align}
If one expresses the cosines and sines via \PMlinkname{Euler formulas}{ComplexSineAndCosine} with \PMlinkname{exponential function}{ComplexExponentialFunction}, the series (1) attains the form
\begin{align}
f(t) = \sum_{n=-\infty}^\infty c_ne^{\frac{in\pi t}{p}}.
\end{align}
The coefficients $c_n$ could be obtained of $a_n$ and $b_n$, but they are comfortably derived directly by multiplying the equation (3) by $e^{-\frac{im\pi t}{p}}$ and integrating it from $-p$ to $p$.\, One obtains
\begin{align}
c_n = \frac{1}{2p}\int_{-p}^{\,p}f(t)e^{\frac{-in\pi t}{p}}\,dt \qquad (n = 0,\,\pm1,\,\pm2,\,\ldots).
\end{align}

We may say that in (3), $f(t)$ has been dissolved to sum of {\em harmonics} (elementary waves) $c_ne^{\frac{in\pi t}{p}}$ with amplitudes $c_n$ corresponding the frequencies $n$.

\subsection{Derivation of Fourier integral}

For seeing how the expansion (3) changes when\, $p \to \infty$,\, we put first the expressions (4) of $c_n$ to the series (3):
$$f(t) = \sum_{n=-\infty}^\infty e^{\frac{in\pi t}{p}}\frac{1}{2p}\int_{-p}^{\,p}f(t)e^{\frac{-in\pi t}{p}}\,dt$$
By denoting\, $\omega_n := \frac{n\pi}{p}$\, and\, $\Delta_n\omega := \omega_{n+1}\!-\!\omega_n = \frac{\pi}{p}$,\, the last equation takes the form
$$f(t) = \frac{1}{2\pi}\sum_{n=-\infty}^\infty e^{i\omega_nt}\Delta_n\omega \int_{-p}^{\,p}f(t)e^{-i\omega_nt}\,dt.$$
It can be shown that when\, $p \to \infty$\, and thus\, $\Delta_n\omega \to 0$,\, the limiting form of this equation is
\begin{align}
f(t) \,=\, \frac{1}{2\pi}\int_{-\infty}^{\,\infty} e^{i\omega t}d\omega\int_{-\infty}^{\,\infty} f(t)e^{-i\omega t}dt.
\end{align}
Here, $f(t)$ has been represented as a {\em Fourier integral}.\, It can be proved that for validity of the expansion (4) it suffices that the function $f$ is piecewise continuous on every finite interval having at most a finite amount of extremum points and that the integral
$$\int_{-\infty}^{\,\infty}|f(t)|\,dt$$
converges.

For better to compare to the Fourier series (3) and the coefficients (4), we can write (5) as
\begin{align}
f(t) \,=\, \int_{-\infty}^{\,\infty}c(\omega)e^{i\omega t}d\omega,
\end{align}
where
\begin{align}
c(\omega) \,=\, \frac{1}{2\pi}\int_{-\infty}^{\,\infty}f(t)e^{-i\omega t}dt.
\end{align}

\subsection{Fourier transform}

If we denote $2\pi c(\omega)$ as
\begin{align}
F(\omega) \,=\, \int_{-\infty}^{\,\infty} e^{-i\omega t}f(t)\,dt,
\end{align}
then by (5),
\begin{align}
f(t) \,=\, \frac{1}{2\pi}\int_{-\infty}^{\,\infty}e^{i\omega t}F(\omega)\,d\omega.
\end{align}
$F(\omega)$ is called the {\em Fourier transform} of $f(t)$.\, It is an integral transform and (9) \PMlinkescapetext{represents} its inverse transform.

N.B. that often one sees both the formula (8) and the formula (9) equipped with the same constant factor $\displaystyle\frac{1}{\sqrt{2\pi}}$ in front of the integral sign.

\begin{thebibliography}{9}
\bibitem{K.V.}{\sc K. V\"ais\"al\"a:} {\em Laplace-muunnos}.\, Handout Nr. 163.\quad Teknillisen korkeakoulun ylioppilaskunta, Otaniemi, Finland (1968).
\end{thebibliography}






%%%%%
%%%%%
\end{document}
