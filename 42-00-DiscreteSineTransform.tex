\documentclass[12pt]{article}
\usepackage{pmmeta}
\pmcanonicalname{DiscreteSineTransform}
\pmcreated{2013-03-22 17:23:45}
\pmmodified{2013-03-22 17:23:45}
\pmowner{stitch}{17269}
\pmmodifier{stitch}{17269}
\pmtitle{discrete sine transform}
\pmrecord{7}{39764}
\pmprivacy{1}
\pmauthor{stitch}{17269}
\pmtype{Definition}
\pmcomment{trigger rebuild}
\pmclassification{msc}{42-00}
\pmclassification{msc}{65T50}
\pmsynonym{DST}{DiscreteSineTransform}
\pmsynonym{discrete trigonometric transforms}{DiscreteSineTransform}
\pmrelated{DiscreteCosineTransform}
\pmrelated{DiscreteFourierTransform2}
\pmrelated{DiscreteFourierTransform}
\pmdefines{DST-I}
\pmdefines{DST-II}
\pmdefines{DST-III}
\pmdefines{DST-IV}
\pmdefines{DST-V}
\pmdefines{DST-VI}
\pmdefines{DST-VII}
\pmdefines{DST-VII}
\pmdefines{DST-VIII}

\endmetadata

% this is the default PlanetMath preamble.  as your knowledge
% of TeX increases, you will probably want to edit this, but
% it should be fine as is for beginners.

% almost certainly you want these
\usepackage{amssymb}
\usepackage{amsmath}
\usepackage{amsfonts}

% used for TeXing text within eps files
%\usepackage{psfrag}
% need this for including graphics (\includegraphics)
%\usepackage{graphicx}
% for neatly defining theorems and propositions
%\usepackage{amsthm}
% making logically defined graphics
%%%\usepackage{xypic}

% there are many more packages, add them here as you need them

% define commands here

\begin{document}
The \PMlinkescapetext{\emph{discrete sine transforms (DST)}} are a family of \PMlinkescapetext{similar} transforms closely related to the discrete cosine transform and the discrete Fourier transform. The \PMlinkescapetext{complete} set of variants of the DST was first introduced by Wang and Hunt \cite{DWT}.

\section{Definition}

The orthonormal variants of the DST, where $x_n$ is the original vector of $N$ real numbers, $C_k$ is the transformed vector of $N$ real numbers and $\delta$ is the Kronecker delta, are defined by the following equations:

\subsection{DST-I}
\begin{eqnarray*}
S^{I}_k&=&p \sum _{n=0}^{N-1} x_n \sin \frac{\pi (n+1) (k+1)}{N+1} \quad \quad k=0, 1, 2, \dots, N-1\\
p&=&\sqrt{\frac{2}{N+1}}
\end{eqnarray*}
The DST-I is its own inverse.

\subsection{DST-II}
\begin{eqnarray*}
S^{II}_k&=&p_k \sum _{n=0}^{N-1} x_n \sin \frac{\pi \left(n+\frac{1}{2}\right) (k+1)}{N} \quad \quad k=0, 1, 2, \dots, N-1\\
p_k&=&\sqrt{\frac{2-\delta _{k,0}}{N}}
\end{eqnarray*}
The inverse of DST-II is DST-III.

\subsection{DST-III}
\begin{eqnarray*}
S^{III}_k&=&p \sum _{n=0}^{N-1} x_n q_n \sin \frac{\pi (n+1) \left(k+\frac{1}{2}\right)}{N} \quad \quad k=0, 1, 2, \dots, N-1\\
p&=&\sqrt{\frac{2}{N}}\\
q_n&=&\sqrt{\frac{1}{1+\delta _{n,0}}}
\end{eqnarray*}
The inverse of DST-III is DST-II.

\subsection{DST-IV}
\begin{eqnarray*}
S^{IV}_k&=&p \sum _{n=0}^{N-1} x_n \sin \frac{\pi \left(n+\frac{1}{2}\right) \left(k+\frac{1}{2}\right)}{N} \quad \quad k=0, 1, 2, \dots, N-1\\
p&=&\sqrt{\frac{2}{N}}
\end{eqnarray*}
The DST-IV is its own inverse.

\subsection{DST-V}
\begin{eqnarray*}
S^V_k&=&p \sum _{n=0}^{N-1} x_n \sin \frac{\pi (n+1) (k+1)}{N+\frac{1}{2}} \quad \quad k=0, 1, 2, \dots, N-1\\
p&=&\sqrt{\frac{2}{N+\frac{1}{2}}}
\end{eqnarray*}
The DST-V is its own inverse.

\subsection{DST-VI}
\begin{eqnarray*}
S^{VI}_k&=&p \sum _{n=0}^{N-1} x_n \sin \frac{\pi \left(n+\frac{1}{2}\right) (k+1)}{N+\frac{1}{2}} \quad \quad k=0, 1, 2, \dots, N-1\\
p&=&\sqrt{\frac{2}{N+\frac{1}{2}}}
\end{eqnarray*}
The inverse of DST-VI is DST-VII.

\subsection{DST-VII}
\begin{eqnarray*}
S^{VII}_k&=&p \sum _{n=0}^{N-1} x_n \sin \frac{\pi (n+1) \left(k+\frac{1}{2}\right)}{N+\frac{1}{2}} \quad \quad k=0, 1, 2, \dots, N-1\\
p&=&\sqrt{\frac{2}{N+\frac{1}{2}}}
\end{eqnarray*}
The inverse of DST-VII is DST-VI.

\subsection{DST-VIII}
\begin{eqnarray*}
S^{VIII}_k&=&p_k \sum _{n=0}^{N-1} x_n q_n \sin \frac{\pi \left(n+\frac{1}{2}\right) \left(k+\frac{1}{2}\right)}{N-\frac{1}{2}} \quad \quad k=0, 1, 2, \dots, N-1\\
p_k&=&\sqrt{\frac{2-\delta _{k,N-1}}{N-\frac{1}{2}}}\\
q_n&=&\sqrt{\frac{1}{1+\delta _{n,N-1}}}
\end{eqnarray*}
The DST-VIII is its own inverse.

\section{Two-dimensional DST}

The DST in two dimensions is simply the one-dimensional transform computed in each row and each column. For example, the DST-II of a $N_1\times N_2$ matrix is given by

\begin{eqnarray*}
S^{II}_{k_1,k_2}&=&p_{k_1}p_{k_2}\sum _{n_1=0}^{N_1-1}\sum _{n_2=0}^{N_2-1} x_{n_1,n_2} \sin \frac{\pi \left(n_1+\frac{1}{2}\right) (k_1+1)}{N_1} \sin \frac{\pi \left(n_2+\frac{1}{2}\right) (k_2+1)}{N_2}
\end{eqnarray*}

\begin{thebibliography}{3}

\bibitem{Shao07} Xuancheng Shao, Steven G. Johnson. Type-II/III DCT/DST algorithms with reduced number of arithmetic operations. 2007.

\bibitem{Pau06} Markus P\"auschel, Jos\'e M. F. Mouray. The algebraic approach to the discrete cosine and
sine transforms and their fast algorithms. 2006.

\bibitem{DWT} Z. Wang and B. Hunt, The Discrete W Transform, Applied Mathematics and Computation,
16. 1985.

\end{thebibliography}
%%%%%
%%%%%
\end{document}
