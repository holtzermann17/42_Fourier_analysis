\documentclass[12pt]{article}
\usepackage{pmmeta}
\pmcanonicalname{ApproximatingFourierIntegralsWithDiscreteFourierTransforms}
\pmcreated{2013-03-22 16:28:23}
\pmmodified{2013-03-22 16:28:23}
\pmowner{stevecheng}{10074}
\pmmodifier{stevecheng}{10074}
\pmtitle{approximating Fourier integrals with discrete Fourier transforms}
\pmrecord{9}{38637}
\pmprivacy{1}
\pmauthor{stevecheng}{10074}
\pmtype{Derivation}
\pmcomment{trigger rebuild}
\pmclassification{msc}{42B10}
\pmclassification{msc}{65T50}
\pmclassification{msc}{42A38}
\pmrelated{DiscreteTimeFourierTransformInRelationWithItsContinousTimeFourierTransfrom}

% The standard font packages
\usepackage{amssymb}
\usepackage{amsmath}
\usepackage{amsfonts}

% For neatly defining theorems and definitions
%\usepackage{amsthm}

% Including EPS/PDF graphics (\includegraphics)
%\usepackage{graphicx}

% Making matrix-based graphics
%%%\usepackage{xypic}

% Enumeration lists with different styles
%\usepackage{enumerate}

% Set up the theorem environments
%\newtheorem{thm}{Theorem}
%\newtheorem*{thm*}{Theorem}

\providecommand{\defnterm}[1]{\emph{#1}}

% The standard number systems
\newcommand{\complex}{\mathbb{C}}
\newcommand{\real}{\mathbb{R}}
\newcommand{\rat}{\mathbb{Q}}
\newcommand{\nat}{\mathbb{N}}
\newcommand{\intset}{\mathbb{Z}}

% Absolute values and norms
% Normal, wide, and big versions of the delimeters
\providecommand{\abs}[1]{\lvert#1\rvert}
\providecommand{\absW}[1]{\left\lvert#1\right\rvert}
\providecommand{\absB}[1]{\Bigl\lvert#1\Bigr\rvert}
\providecommand{\norm}[1]{\lVert#1\rVert}
\providecommand{\normW}[1]{\left\lVert#1\right\rVert}
\providecommand{\normB}[1]{\Bigl\lVert#1\Bigr\rVert}

% Differentiation operators
\providecommand{\od}[2]{\frac{d #1}{d #2}}
\providecommand{\pd}[2]{\frac{\partial #1}{\partial #2}}
\providecommand{\pdd}[2]{\frac{\partial^2 #1}{\partial #2}}
\providecommand{\ipd}[2]{\partial #1 / \partial #2}

% Differentials on integrals
\newcommand{\dx}{\, dx}
\newcommand{\dt}{\, dt}
\newcommand{\dmu}{\, d\mu}

% Inner products
\providecommand{\ip}[2]{\langle {#1}, {#2} \rangle}

% Calligraphic letters
\newcommand{\sF}{\mathcal{F}}
\newcommand{\sD}{\mathcal{D}}

% Standard spaces
\newcommand{\Hilb}{\mathcal{H}}
\newcommand{\Le}{\mathbf{L}}

% Operators and functions occassionally used in my articles
\DeclareMathOperator{\D}{D}
\DeclareMathOperator{\linspan}{span}
\DeclareMathOperator{\rank}{rank}
\DeclareMathOperator{\lindim}{dim}
\DeclareMathOperator{\sinc}{sinc}

% Probability stuff
\newcommand{\PP}{\mathbb{P}}
\newcommand{\E}{\mathbb{E}}

\providecommand{\ipB}[2]{\biggl\langle {#1}\,, {#2} \biggr\rangle}
\DeclareMathOperator{\Vol}{Vol}
\providecommand{\conjug}[1]{\overline{#1}}

\begin{document}
\PMlinkescapeword{formula}
\PMlinkescapeword{size}
\PMlinkescapeword{components}

Suppose we have a continuous function $f \colon \real^\nu \to \complex$
for which we want to numerically 
compute the continuous Fourier transform:
\[
\hat{f}(\xi) = \int_{\real^\nu} f(x) \, e^{-2\pi i \xi \cdot x} \, dx\,, 
\quad \xi \in \real^\nu\,.
\]
Assume that either the function $f$ is supported
inside the compact rectangle
 $C = [a_1, b_1] \times \dotsb \times [a_\nu, b_\nu]$,
or that the Fourier integral is well approximated
by truncating the domain to $C$.

We develop an approximation to $\hat{f}$ 
based on the discrete Fourier transform
(which can be computed efficiently using the fast Fourier transform
algorithm), and analyze the error in making this approximation.

\subsection*{Derivation}

Performing the linear substitution
\[
t_j = \frac{x - a_j}{b_j - a_j}\,, \quad j = 1, \dotsc, \nu\,,
\]
we find that 
\begin{align*}
&\int_C f(x) \, e^{-2\pi i \xi \cdot x } \, dx\\
=& \Vol(C) \,  e^{-2\pi i \xi \cdot a}
\int_0^1 \dotsb \int_0^1
g(t) \,
e^{-2\pi i \bigl( (b_1 - a_1) \xi_1 t_1 + \dotsb + (b_\nu - a_\nu) \xi_\nu t_\nu \bigr)}\, dt_1 \dotsb dt_\nu \\
=& \Vol(C) \, e^{-2\pi i \xi \cdot a} \,
\hat{g}\bigl((b_1 - a_1)\xi_1, \dotsc, (b_\nu - a_\nu) \xi_\nu\bigr)\,,
\end{align*}
where
\[
g(t_1, \dotsc, t_\nu) = f(a_1 + (b_1 - a_1) t_1, \dotsc, a_\nu + (b_\nu - a_\nu) t_\nu)\,,
\quad
t_j \in [0, 1]\,,
\]
and $\hat{g}$ denotes the Fourier transform of $g$ on the unit cube $[0,1]^\nu$.

Let $u$ denote the vector with components
$u(k_1, \dotsc, k_\nu) = g(k_1/N, \dotsc, k_\nu/N)$, for $k_j = 0, \dotsc, N-1$.
The discrete Fourier transform
of $u$ is
\[
\hat{u}(l) = \sum_{k \in \{0, \dotsc, N-1\}^\nu} u(k) \, e^{-2\pi i k \cdot l/N}\,, \quad l = \{0, \dotsc, N-1\}^\nu\,,
\]
and approximates $N^\nu \hat{g}(l)$
since it is a Riemann sum (step size $1/N$) for the Fourier integral.

Therefore, for $l_j = 0, \dotsc, N-1$,
we have the approximation:
\[
\hat{f}\left( \frac{l_1}{b_1 - a_1}, \dotsc, \frac{l_\nu}{b_\nu - a_\nu} \right)
\approx \frac{\Vol(C)}{N^\nu} \, e^{-2\pi i \left( \frac{a_1 l_1}{b_1 - a_1} + \dotsb + \frac{a_\nu l_\nu}{b_\nu - a_\nu} \right)} \, \hat{u}(l)\,.
\]

\subsection*{Analysing the error}

Assume that $g$ can be represented by a pointwise-convergent\footnote{
Since the discrete Fourier transform requires sampling of points, $\Le^2$
convergence of the Fourier series is not sufficient. Also, the Fourier series
may only be conditionally convergent.  If there are difficulties in assuring
its convergence, Fej\'er summation can be
used instead in this analysis.}
 Fourier series:
\[
g(t) = \sum_{k \in \intset^\nu} \ip{g}{E_k} \, E_k(t)\,, 
\]
where
\[
E_k(t) = e^{2\pi i k \cdot t}\,, \quad
\ip{\alpha}{\beta} = \int_{[0,1]^\nu} \alpha(t) \, \conjug{\beta(t)} \, dt\,.
\]

The Fourier coefficient of $g$ is 
$\hat{g}(l) = \ip{g}{E_l}$, and in contrast 
$\hat{u}(l) = N^\nu \, \ip{g}{E_k}_N$, where:
\[
\ip{\alpha}{\beta}_N = \frac{1}{N^\nu} 
\sum_{k \in \{0, \dotsc, N-1\}^\nu} \alpha(k/N) \, \conjug{\beta(k/N)}
\] 
is a positive semi-definite inner product.

We have the exact formula:
\[
\hat{f}\left( \frac{l_1}{b_1 - a_1}, \dotsc, \frac{l_\nu}{b_\nu - a_\nu} \right)
= \Vol(C) \, e^{-2\pi i \left( \frac{a_1 l_1}{b_1 - a_1} + \dotsb + \frac{a_\nu l_\nu}{b_\nu - a_\nu} \right)} \, \hat{g}(l)\,;
\]
and we wish to know what error is introduced
if we were to replace $\hat{g}(l) = \ip{g}{E_l}$
by the discrete version of the inner product $\ip{g}{E_l}_N = \hat{u}(l)/N^\nu$.

Expand $\ip{g}{E_l}_N$ as:
\[
\ip{g}{E_l}_N = \ipB{ 
\sum_{k \in \intset^\nu} \ip{g}{E_k} E_k}{E_l}_N
= \sum_{k \in \intset^\nu} \ip{g}{E_k} \ip{E_k}{E_l}_N \,.
\]
Since $\ip{E_k}{E_l}_N$ is $1$ if $k_j \equiv l_j \pmod N$ for all $j$,
and is $0$ otherwise,
the discrete inner product reduces to:
\[
\ip{g}{E_l}_N 
= \sum_{k \in \intset^\nu} \ip{g}{E_{l + kN}} 
= \ip{g}{E_l} + \sum_{k \in \intset^\nu \setminus \{0\} } \ip{g}{E_{l + kN}} \,.
\]

In other words, the approximation entails replacing
the true Fourier coefficient $\hat{g}(l)$
with the same coefficient
plus other Fourier coefficients $\ip{g}{E_{l+kN}}$
corresponding to higher
frequencies, in multiples of $N$.
The magnitude of the approximation error
for $\hat{f}$
thus depends on how fast the partial sums
of the Fourier coefficients $\ip{g}{E_k}$ decay to zero.

%%%%%
%%%%%
\end{document}
