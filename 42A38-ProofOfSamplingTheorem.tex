\documentclass[12pt]{article}
\usepackage{pmmeta}
\pmcanonicalname{ProofOfSamplingTheorem}
\pmcreated{2013-03-22 16:28:57}
\pmmodified{2013-03-22 16:28:57}
\pmowner{stevecheng}{10074}
\pmmodifier{stevecheng}{10074}
\pmtitle{proof of sampling theorem}
\pmrecord{13}{38650}
\pmprivacy{1}
\pmauthor{stevecheng}{10074}
\pmtype{Proof}
\pmcomment{trigger rebuild}
\pmclassification{msc}{42A38}
\pmclassification{msc}{94A20}
\pmrelated{PlancherelsTheorem}

\usepackage{amssymb}
\usepackage{amsmath}
\usepackage{amsfonts}
%\usepackage{amsthm}
\usepackage{enumerate}
\usepackage{graphicx}
%\usepackage{psfrag}
%%%\usepackage{xypic}

% define commands here
\newcommand{\complex}{\mathbb{C}}
\newcommand{\real}{\mathbb{R}}
\newcommand{\rat}{\mathbb{Q}}
\newcommand{\nat}{\mathbb{N}}
\newcommand{\intset}{\mathbb{Z}}
\newcommand{\Hilb}{\mathcal{H}}
\newcommand{\FT}{\mathcal{F}}
\newcommand{\Le}{\mathbf{L}}

\providecommand{\abs}[1]{\lvert#1\rvert}
\providecommand{\absW}[1]{\left\lvert#1\right\rvert}
\providecommand{\absB}[1]{\Bigl\lvert#1\Bigr\rvert}
\providecommand{\norm}[1]{\lVert#1\rVert}
\providecommand{\normW}[1]{\left\lVert#1\right\rVert}
\providecommand{\normB}[1]{\Bigl\lVert#1\Bigr\rVert}
\providecommand{\defnterm}[1]{\emph{#1}}

\DeclareMathOperator{\sinc}{sinc}
\begin{document}
\PMlinkescapeword{sides}
\PMlinkescapeword{side}
\PMlinkescapeword{length}
\PMlinkescapeword{estimate}
\PMlinkescapeword{representation}

\subsection{Set-up}

Let $w > 0$ be the (two-sided) bandwidth.  The variable $\xi$ below will denote
frequency, and the variable $t$ will denote time.
(Both $w$ and $\xi$ are measured in \PMlinkescapetext{cycles per unit time}.)

Consider the space of functions:
\[
\Hilb' = \{ g \in \Le^2(\real) \colon g(\xi) = 0 \text{ for almost all } \abs{\xi} > w/2 \}
\]
which is clearly seen to be a complex Hilbert space with the usual inner product for $\Le^2(\real)$.

Let $\FT$ denote the Fourier transform on $\Le^2(\real)$,
which is a unitary transform by Plancherel's theorem.
So,
\[
\Hilb = \{ f \in \Le^2(\real) \colon 
(\FT f)(\xi) = 0 \text{ for almost all } \abs{\xi} > w/2 \} 
= \FT^{-1} \Hilb'
\]
is also a Hilbert space.

\subsection{Computation of orthonormal basis}

One orthonormal basis for $\Hilb'$ consists of the usual Fourier \PMlinkescapetext{basis} functions
on the interval $[-w/2, w/2]$, extended to be zero on $\real \setminus [-w/2, w/2]$:
\[
\phi_n(\xi) = \begin{cases}
\frac{1}{\sqrt{w}} e^{-2 \pi i n \xi/w}\,, & \abs{\xi} \leq w/2 \\
0\,, & \abs{\xi} > w/2\,,
\end{cases}
\quad
n \in \intset\,.
\]
Mapping these by $\FT^{-1}$ produces an orthonormal basis for $\Hilb$:
\begin{align*}
(\FT^{-1} \phi_n)(t) &= \frac{1}{\sqrt{w}} \int_{-w/2}^{w/2} e^{-2 \pi i n \xi/w}
\, e^{2\pi i \xi t} \, d\xi \\
&= \frac{1}{\sqrt{w}} \int_{-w/2}^{w/2} e^{2 \pi i \xi (t-n/w)} \, d\xi \\
&= \frac{1}{\sqrt{w}} \bigl( w \, \sinc(w (t-n/w)) \bigr) = \sqrt{w} \, \sinc(wt-n)\,,
\end{align*}
where we have used the fact that the Fourier transform of $t \mapsto w \, \sinc(wt)$ (normalized sinc function)
is the rectangular box function of bandwidth $w$, and vice versa.

\subsection{Expansion by orthonormal basis}

Given $f \in \Hilb$, let $g = \FT f \in \Hilb'$.
We can expand $g$ in a Fourier series with respect
to the \PMlinkescapetext{basis} $\{ \phi_n \}$:
\[
g(\xi) = \sum_{n \in \intset} \langle g, \phi_n \rangle \, \phi_n(\xi)\,,
\]
with the infinite sum converging in $\Le^2(\real)$.
Taking $\FT^{-1}$ of both sides,
we obtain:
\[
f(t) = (\FT^{-1} g)(t) = \sum_{n \in \intset}
\langle g, \phi_n \rangle \, (\FT^{-1} \phi_n)(t) 
=
\sqrt{w} \sum_{n \in \intset}
\langle g, \phi_n \rangle \, \sinc(wt-n)\,.
\]
Moreover,
\[
\langle g, \phi_n \rangle =
\frac{1}{\sqrt{w}} \int_{-\infty}^\infty g(\xi) \, e^{2\pi i n \xi/w} \, d\xi
= \frac{1}{\sqrt{w}} (\FT^{-1} g)\Bigl( \frac{n}{w} \Bigr) = \frac{1}{\sqrt{w}} f\Bigl( \frac{n}{w} \Bigr)\,.
\]
(Since $g$ is also in $\Le^1$, its inverse Fourier transform
$\FT^{-1}g$ is a continuous function.  
Provided that we modify
$f$ on a set of measure zero, we can assume that $f = \FT^{-1} (\FT f) = \FT^{-1} g$ is continuous.  So it is legal to talk about the pointwise
values $f(n/w)$.)

\subsection{Result}

Hence, we arrive at the representation:
\[
f(t) = \sum_{n \in \intset} f\Bigl( \frac{n}{w} \Bigr) \, \sinc(wt - n)\,,
\]
thereby reconstructing any $f \in \Hilb$ --- a square-integrable band-limited 
function ---
from its samples at every time period of length $1/w$.

\subsection{Uniform and absolute convergence of series}

The infinite series for $f$ converges in $\Le^2$
by construction, but in fact it also converges uniformly and absolutely.
To see this, first note that by the Cauchy-Schwarz inequality,
\[
\sum_{n \in \intset} \abs{ f(\tfrac{n}{w}) }
\, \abs{\sinc (wt-n)}
\leq \Bigl( \sum_{n \in \intset} \abs{f(\tfrac{n}{w})}^2 \Bigr)^{1/2}
\Bigl( \sum_{n \in \intset} \sinc^2(wt -n) \Bigr)^{1/2}\,.
\]
The series $\sum_n \abs{f(n/w)}^2$ converges by Parseval's theorem
($w^{-1/2} f(n/w)$ are the Fourier coefficients of $g$).
Also, the series $\sum_n \sinc^2(wt-n)$ is uniformly bounded for all $t \in \real$.  To prove this, it suffices to 
restrict to $t$ bounded inside $[0, 1/w]$
as the function $t \mapsto \sum_n \sinc^2(wt-n)$ is $1/w$-periodic; and then 
it becomes an easy estimate using the fact that $\sum_n n^{-2} < \infty$.
It follows that the series $\sum_n \abs{f(n/w)} \abs{\sinc(wt-n)}$
is uniformly bounded for all $t$,
and its tail tends to zero uniformly in $t$.

\subsection{Illustrations}

\begin{figure}[!htb]
\begin{center}
\includegraphics{bessel.eps}
\caption{Reconstructing a Bessel function (of the first kind), using a $\sinc$ series with $n = -10, -9 \dotsc, 0, \dotsc, 10$.  Theoretically exact bandwidth is $w = 1/\pi \approx 0.32$.}
\end{center}
\end{figure}

\begin{figure}[!htb]
\begin{center}
\includegraphics{bessel-extra.eps}
\caption{Effect of under-sampling and over-sampling beyond the actual bandwidth of the function}
\end{center}
\end{figure}

\begin{figure}[!htb]
\begin{center}
\includegraphics{basis.eps}
\caption{The basis functions $\sinc(t)$, $\tfrac12 \sinc(t\pm 1)$, $\tfrac14 \sinc(t \pm 2)$}
\end{center}
\end{figure}

\begin{itemize}
\item
\PMlinkexternal{Python program to produce the three figures}{http://aux.planetmath.org/files/objects/8650/sample.py}
\end{itemize}

%%%%%
%%%%%
\end{document}
